\documentclass[a4paper]{book}
\usepackage[russian]{babel}
\usepackage[utf8]{inputenc}
\usepackage[T2A]{fontenc}
\usepackage{graphicx}

\usepackage{amsmath,amsthm}
\newtheoremstyle{problemstyle}  % <name>
        {3pt}                 % <space above>
        {3pt}                 % <space below>
        {\normalfont}         % <body font>
        {}                    % <indent amount}
        {\normalfont\bfseries}   % <theorem head font>
        {\normalfont\bfseries.} % <punctuation after theorem head>
        {.5em}              % <space after theorem head>
        {}  % <theorem head spec (can be left empty, meaning `normal')>
\theoremstyle{problemstyle}
\newtheorem{problem}{} % нумерация задач задумана сквозной

\newcommand{\divisible}{\mathop{\raisebox{-2pt}{\vdots}}}

\usepackage{hyperref}

\author{Трус \and Балбес \and Бывалый}
\title{Задачник для детей от 15 до 25}
\begin{document}
    \maketitle
    \input{0-part}
    \tableofcontents
        \chapter{Разминка}
    \begin{problem}
        Раскрасить поверхность бутылки Клейна различными цветами так, чтобы каждая область определённого цвета
        граничила со всеми остальными. Какое минимальное число цветов необходимо?\\
        \textit{Примечание: точка не является границей двух областей.}
    \end{problem}   
    \begin{problem}
        Что получится, если вывернуть через отверстие тор n-го рода в k - мерном евклидовом пространстве?
    \end{problem} 
    \begin{problem}
        Вычислить \int \vec{r} d\vec{S} по поверхности дикой сферы. Обобщить на n-мерный случай.
    \end{problem} 
    \includegraphics{sphr}
    \chapter{Геометрия}
    \begin{problem}
        Определить объём n-мерной квадратной пирамиды.
    \end{problem}    
    \begin{problem}
        Определить объём n-мерного аналога тетраэдра для всех \( n \ge 3 \).
    \end{problem}
    \begin{problem}
        Исследовать зависимость отношения объёма гиперсферы к объёму описанного
        около неё гиперкуба от размерности евклидова пространства.
    \end{problem}
    \chapter{Анализ}
    \begin{problem}
        Имеется сферическое однородное облако плазмы, которое покоится в
        начальный момент. Пусть его радиус \( R_0 \), а заряд \( Q_0 \).
        Определить характер его движения и получить зависимость плотности заряда
        от координат и времени \( \rho(\vec{r}, t) \).
    \end{problem}
    \begin{problem}
        Определить закон движения <<солнечного паруса>> (идеально отражающего
        зеркала) площадью \( S \) и массой \(m\) в поле тяготения Солнца.
    \end{problem}
    \begin{problem}
        Как будет выглядеть формула Циолковского для фотонной ракеты, двигатель
        которой выполнен в виде параболического зеркала, в фокусе которого
        происходит аннигиляция горючего из вещества и антивещества?
    \end{problem}
    \chapter{Комбинаторика}
    \begin{problem}
        <<Счастливым>> называется билет с 6-значным номером, сумма первых трёх
        цифр которого равна сумме последних трёх\footnote{в Волгограде такие
        билеты называют <<счастливыми по-московски>>; существуют также билеты,
        <<счастливые по-питерски>>}. Сколько всего существует
        <<счастливых>> билетов?
    \end{problem}
    \begin{problem}
        Возможно ли расставить по кругу числа от 1 до 12 так, чтобы для любых 3
        последовательно расположенных чисел \(a\), \(b\) и \(c\) выполнялось
        условие \( b^2 - ac \divisible 13 \)? Если да, то сколько различных
        способов (без учёта начального элемента и направления обхода) существует?
    \end{problem}
\end{document}
